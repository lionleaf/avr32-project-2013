This assignment’s stated task was to create an assembly program for the STK1000 that would turn on the central LED and allow a user to shift the light either left or right by pressing buttons that are read in an interrupt routine. Specifically, this required the light-switching to be placed in the main loop.

There was no formal test-environment for this assignment and the main manner of testing was to visually see if the buttons and LEDs behaved as they should. This meant whether the correct lights were turned on and on time, and were turned off correctly, and on time. 
Eventually the group-members were satisfied with the resulting program’s response time and behaviour in terms of the assignment’s requirements. At that point the program was deemed finished.

\subsection{Tests}
Vent litt, lurer på om vi skal dele dem opp i mindre tester? Hvertfall ha en egen test for “loopingen”. sure

Bare for å huske nummerering på LEDs og knapper:
7-6-5-4-3-2-1-0
Test ideas:
Start led to the right, press left button until it reaches the leftmost location and vice versa
Start LED at the leftmost position and press the left button to validate looping and vice versa.
Push no buttons (LED is to stay in the same position)
Press buttons slowly (check debouncing)
Press buttons hard?? (or is this the worst bounce condition?)
Tror ikke vi trenger å ta hensyn til måten vi trykker på knappene. Var tenkt som en spesifik test for å sjekke vår debouncing. Kan finne på å skape bugs, altså bra test? Men vet ikke helt hvordan man gjør debouncing verst, så kan droppe det.
Press button x times reasonably fast and check wether it moved x spaces. Used to validate that our debouncing isn’t overly long

Ok, hvordan vil vi at det skal se ut i LaTeX? Lister som her?







\begin{tabular}[h]{lp{12cm}} \hline
\textbf{\emph{Test:}} 		& \textbf{Move right}\\
\emph{Action:} 		& Press button 0\\
\emph{Preconditions:}	& One, and only one, LED lit. The number of the lit LED should be between 1 and 7.\\
\emph{Wanted outcome:}	& One, and only one, LED should be lit all the time. After the button press, the LED right of the currently lit should turn on, and the previously lit should turn off.\\ \hline
\end{tabular}


\begin{tabular}[h]{lp{12cm}} \hline
\textbf{\emph{Test:}} 		& \textbf{Move left}\\
\emph{Action:} 		& Press button 2\\
\emph{Preconditions:}	& One, and only one, LED lit. The number of the lit LED should be between 0 and 6.\\
\emph{Wanted outcome:}	& One, and only one, LED should be lit all the time. After the button press, the LED left of the currently lit should turn on, and the previously lit should turn off. \\ \hline
\end{tabular}

\begin{tabular}[h]{lp{12cm}} \hline
\textbf{\emph{Test:}} 		& \textbf{Wrap right}\\
\emph{Action:} 		& Press button 0\\
\emph{Preconditions:}	& LED 0 lit and all others unlit.\\
\emph{Wanted outcome:}	& After the button press, LED 0 should turn off and LED 7 should turn on.\\ \hline
\end{tabular}

\begin{tabular}[h]{lp{12cm}} \hline
\textbf{\emph{Test:}} 		& \textbf{Wrap left}\\
\emph{Action:} 		& Press button 2\\
\emph{Preconditions:}	& LED 7 lit and all others unlit.\\
\emph{Wanted outcome:}	& After the button press, LED 7 should turn off and LED 0 should turn on.\\ \hline
\end{tabular}


